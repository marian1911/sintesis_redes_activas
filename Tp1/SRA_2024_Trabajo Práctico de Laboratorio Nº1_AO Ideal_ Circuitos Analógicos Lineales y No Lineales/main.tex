\documentclass{article}
\usepackage{import}
\usepackage{example}
\usepackage{Estilo_Relatorio}
\usepackage{siunitx}
\usepackage{steinmetz}
\usepackage{enumerate}
\usepackage{verbatim}

\headerp{Cátedra de Síntesis de Redes Activas - Trabajos Prácticos de Laboratorio}
 
\titulo{Trabajo Práctico de Laboratorio Nº1\\
AO Ideal: Circuitos Analógicos Lineales y No Lineales.}
\data{Año 2024}
% \autor{nombre}
\author{
\large
%\textsc{\textbf{Alumno:} asde}
\textsc{\textbf{Profesor Titular:} Dr. Ing. Pablo Ferreyra}\\
\textsc{\textbf{Profesor Adjunto:} Ing. César Reale} \\
\textsc{\textbf{Profesor Ayudate:} TBD} \\
\textsc{\textbf{Ayudante alumno:} TBD} \\
\normalsize Facultad de Ciencias Exactas Físicas y Naturales \\ Universidad Nacional de Córdoba  % Your institution
\vspace{-5mm}
}


\begin{document}

\maketitle

\begin{center}

\textbf{Síntesis de Redes Activas}\\

\textbf{Ingeniería Electrónica - Agosto 2024}\\

\end{center}

\thispagestyle{fancy}

%%%%%%%%%%%%%%%%%%%%%%%%%%%%%%%%%%%%%%%%%%%%%%%%%%

 \begin{abstract}

 \noindent Primer laboratorio cuyo objetivo es familiarizarse con el armado y análisis de circuitos analógicos lineales y no lineales. En este
Trabajo Práctico debe considerar para los cálculos iniciales el amplificador como ideal. .

 \end{abstract}

\section{Metodología general}
\import{}{0-Metodologia}

\newpage{}
\section{Circuito I: Amplificador diferencial}
\import{}{1-CI}
\newpage
\section{Circuito II: Fuente de Corriente Controlada por tensión}
\import{}{2-CII}

\newpage
\section{Circuito III: Rectificador de precisión}
\import{}{3-CIII}

\newpage{}
\section{Circuito IV: Comparador con histéresis}
\import{}{4-CIV}

\newpage{}
\section{Ejercicio adicional I}
Diseñar un regulador de carga de batería, que corte cuando se alcanzan los 12.8V y reinicie la carga cuando baja a 10.5V.

\textbf{MATERIALES:}
\begin{itemize}
    \item AO ideal con saturación.
    \item Resistencias
 \item 1 Rele 12V, Normal Abierto, 20mA de corriente de bobina.
 \item 1 Transistor NPN B548 o 1 Transistor PNP BC558.
 \item 1 Diodo 1N4148
 \item 1 Referencia de Tensión: TL431
 \item Batería 12V (Rango 8V a 13V) – $R_{interna}=0.5\Omega$
 \item Celda Fotovoltaica: 15V Tensión Sin Carga, 1A de Corriente de Carga
\end{itemize}

\section{Ejercicio adicional II}

Diseñar un oscilador de relajación que oscile a 1kHz.

\textbf{MATERIALES:}
\begin{itemize}
\item AO ideal con saturación. Vcc=10V Vss=-10V
\item Resistencias
\item Capacitor de 1uF

\end{itemize}
\end{document}

